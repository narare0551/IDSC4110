% Options for packages loaded elsewhere
\PassOptionsToPackage{unicode}{hyperref}
\PassOptionsToPackage{hyphens}{url}
%
\documentclass[
]{article}
\usepackage{amsmath,amssymb}
\usepackage{lmodern}
\usepackage{ifxetex,ifluatex}
\ifnum 0\ifxetex 1\fi\ifluatex 1\fi=0 % if pdftex
  \usepackage[T1]{fontenc}
  \usepackage[utf8]{inputenc}
  \usepackage{textcomp} % provide euro and other symbols
\else % if luatex or xetex
  \usepackage{unicode-math}
  \defaultfontfeatures{Scale=MatchLowercase}
  \defaultfontfeatures[\rmfamily]{Ligatures=TeX,Scale=1}
\fi
% Use upquote if available, for straight quotes in verbatim environments
\IfFileExists{upquote.sty}{\usepackage{upquote}}{}
\IfFileExists{microtype.sty}{% use microtype if available
  \usepackage[]{microtype}
  \UseMicrotypeSet[protrusion]{basicmath} % disable protrusion for tt fonts
}{}
\makeatletter
\@ifundefined{KOMAClassName}{% if non-KOMA class
  \IfFileExists{parskip.sty}{%
    \usepackage{parskip}
  }{% else
    \setlength{\parindent}{0pt}
    \setlength{\parskip}{6pt plus 2pt minus 1pt}}
}{% if KOMA class
  \KOMAoptions{parskip=half}}
\makeatother
\usepackage{xcolor}
\IfFileExists{xurl.sty}{\usepackage{xurl}}{} % add URL line breaks if available
\IfFileExists{bookmark.sty}{\usepackage{bookmark}}{\usepackage{hyperref}}
\hypersetup{
  pdftitle={Struct1\_DataFrames.r},
  pdfauthor={narae},
  hidelinks,
  pdfcreator={LaTeX via pandoc}}
\urlstyle{same} % disable monospaced font for URLs
\usepackage[margin=1in]{geometry}
\usepackage{color}
\usepackage{fancyvrb}
\newcommand{\VerbBar}{|}
\newcommand{\VERB}{\Verb[commandchars=\\\{\}]}
\DefineVerbatimEnvironment{Highlighting}{Verbatim}{commandchars=\\\{\}}
% Add ',fontsize=\small' for more characters per line
\usepackage{framed}
\definecolor{shadecolor}{RGB}{248,248,248}
\newenvironment{Shaded}{\begin{snugshade}}{\end{snugshade}}
\newcommand{\AlertTok}[1]{\textcolor[rgb]{0.94,0.16,0.16}{#1}}
\newcommand{\AnnotationTok}[1]{\textcolor[rgb]{0.56,0.35,0.01}{\textbf{\textit{#1}}}}
\newcommand{\AttributeTok}[1]{\textcolor[rgb]{0.77,0.63,0.00}{#1}}
\newcommand{\BaseNTok}[1]{\textcolor[rgb]{0.00,0.00,0.81}{#1}}
\newcommand{\BuiltInTok}[1]{#1}
\newcommand{\CharTok}[1]{\textcolor[rgb]{0.31,0.60,0.02}{#1}}
\newcommand{\CommentTok}[1]{\textcolor[rgb]{0.56,0.35,0.01}{\textit{#1}}}
\newcommand{\CommentVarTok}[1]{\textcolor[rgb]{0.56,0.35,0.01}{\textbf{\textit{#1}}}}
\newcommand{\ConstantTok}[1]{\textcolor[rgb]{0.00,0.00,0.00}{#1}}
\newcommand{\ControlFlowTok}[1]{\textcolor[rgb]{0.13,0.29,0.53}{\textbf{#1}}}
\newcommand{\DataTypeTok}[1]{\textcolor[rgb]{0.13,0.29,0.53}{#1}}
\newcommand{\DecValTok}[1]{\textcolor[rgb]{0.00,0.00,0.81}{#1}}
\newcommand{\DocumentationTok}[1]{\textcolor[rgb]{0.56,0.35,0.01}{\textbf{\textit{#1}}}}
\newcommand{\ErrorTok}[1]{\textcolor[rgb]{0.64,0.00,0.00}{\textbf{#1}}}
\newcommand{\ExtensionTok}[1]{#1}
\newcommand{\FloatTok}[1]{\textcolor[rgb]{0.00,0.00,0.81}{#1}}
\newcommand{\FunctionTok}[1]{\textcolor[rgb]{0.00,0.00,0.00}{#1}}
\newcommand{\ImportTok}[1]{#1}
\newcommand{\InformationTok}[1]{\textcolor[rgb]{0.56,0.35,0.01}{\textbf{\textit{#1}}}}
\newcommand{\KeywordTok}[1]{\textcolor[rgb]{0.13,0.29,0.53}{\textbf{#1}}}
\newcommand{\NormalTok}[1]{#1}
\newcommand{\OperatorTok}[1]{\textcolor[rgb]{0.81,0.36,0.00}{\textbf{#1}}}
\newcommand{\OtherTok}[1]{\textcolor[rgb]{0.56,0.35,0.01}{#1}}
\newcommand{\PreprocessorTok}[1]{\textcolor[rgb]{0.56,0.35,0.01}{\textit{#1}}}
\newcommand{\RegionMarkerTok}[1]{#1}
\newcommand{\SpecialCharTok}[1]{\textcolor[rgb]{0.00,0.00,0.00}{#1}}
\newcommand{\SpecialStringTok}[1]{\textcolor[rgb]{0.31,0.60,0.02}{#1}}
\newcommand{\StringTok}[1]{\textcolor[rgb]{0.31,0.60,0.02}{#1}}
\newcommand{\VariableTok}[1]{\textcolor[rgb]{0.00,0.00,0.00}{#1}}
\newcommand{\VerbatimStringTok}[1]{\textcolor[rgb]{0.31,0.60,0.02}{#1}}
\newcommand{\WarningTok}[1]{\textcolor[rgb]{0.56,0.35,0.01}{\textbf{\textit{#1}}}}
\usepackage{graphicx}
\makeatletter
\def\maxwidth{\ifdim\Gin@nat@width>\linewidth\linewidth\else\Gin@nat@width\fi}
\def\maxheight{\ifdim\Gin@nat@height>\textheight\textheight\else\Gin@nat@height\fi}
\makeatother
% Scale images if necessary, so that they will not overflow the page
% margins by default, and it is still possible to overwrite the defaults
% using explicit options in \includegraphics[width, height, ...]{}
\setkeys{Gin}{width=\maxwidth,height=\maxheight,keepaspectratio}
% Set default figure placement to htbp
\makeatletter
\def\fps@figure{htbp}
\makeatother
\setlength{\emergencystretch}{3em} % prevent overfull lines
\providecommand{\tightlist}{%
  \setlength{\itemsep}{0pt}\setlength{\parskip}{0pt}}
\setcounter{secnumdepth}{-\maxdimen} % remove section numbering
\ifluatex
  \usepackage{selnolig}  % disable illegal ligatures
\fi

\title{Struct1\_DataFrames.r}
\author{narae}
\date{2021-11-01}

\begin{document}
\maketitle

\begin{Shaded}
\begin{Highlighting}[]
\CommentTok{\# ******************************************************************************}
\CommentTok{\# *** Introduction to R: Data Structures ***************************************}
\CommentTok{\# ******************************************************************************}

\CommentTok{\# ******************************************************************************}
\CommentTok{\# Slide\#4: Data Frames {-} Construction}
\CommentTok{\# ******************************************************************************}
\CommentTok{\# Create data frame of financial loans using interest rates and loan types}
\CommentTok{\# as well as amounts borrowed, loan terms and monthly payment vectors}
\NormalTok{amount }\OtherTok{\textless{}{-}} \FunctionTok{c}\NormalTok{(}\DecValTok{200000}\NormalTok{,}\DecValTok{150000}\NormalTok{,}\DecValTok{100000}\NormalTok{,}\DecValTok{25000}\NormalTok{,}\DecValTok{10000}\NormalTok{,}\DecValTok{200000}\NormalTok{,}\DecValTok{15000}\NormalTok{,}\DecValTok{25000}\NormalTok{,}\DecValTok{150000}\NormalTok{,}\DecValTok{475000}\NormalTok{)}
\NormalTok{intRate }\OtherTok{\textless{}{-}} \FunctionTok{c}\NormalTok{(}\FloatTok{0.07}\NormalTok{,}\FloatTok{0.075}\NormalTok{,}\FloatTok{0.07}\NormalTok{,}\FloatTok{0.065}\NormalTok{,}\FloatTok{0.077}\NormalTok{,}\FloatTok{0.0625}\NormalTok{,}\FloatTok{0.065}\NormalTok{,}\FloatTok{0.0775}\NormalTok{,}\FloatTok{0.0575}\NormalTok{,}\FloatTok{0.0575}\NormalTok{)}
\NormalTok{loanTerm }\OtherTok{\textless{}{-}} \FunctionTok{c}\NormalTok{(}\DecValTok{15}\NormalTok{,}\DecValTok{15}\NormalTok{,}\DecValTok{30}\NormalTok{,}\DecValTok{3}\NormalTok{,}\DecValTok{5}\NormalTok{,}\DecValTok{15}\NormalTok{,}\DecValTok{3}\NormalTok{,}\DecValTok{4}\NormalTok{,}\DecValTok{15}\NormalTok{,}\DecValTok{15}\NormalTok{)}
\NormalTok{loanType }\OtherTok{\textless{}{-}} \FunctionTok{c}\NormalTok{(}\StringTok{\textquotesingle{}Mortg\textquotesingle{}}\NormalTok{,}\StringTok{\textquotesingle{}Mortg\textquotesingle{}}\NormalTok{,}\StringTok{\textquotesingle{}Mortg\textquotesingle{}}\NormalTok{,}\StringTok{\textquotesingle{}Car\textquotesingle{}}\NormalTok{,}\StringTok{\textquotesingle{}Car\textquotesingle{}}\NormalTok{,}\StringTok{\textquotesingle{}Mortg\textquotesingle{}}\NormalTok{,}\StringTok{\textquotesingle{}Other\textquotesingle{}}\NormalTok{,}\StringTok{\textquotesingle{}Car\textquotesingle{}}\NormalTok{,}\StringTok{\textquotesingle{}Mortg\textquotesingle{}}\NormalTok{,}\StringTok{\textquotesingle{}Mortg\textquotesingle{}}\NormalTok{)}
\NormalTok{mthPmt }\OtherTok{\textless{}{-}} \FunctionTok{c}\NormalTok{(}\FloatTok{1787.23}\NormalTok{,}\FloatTok{1381.88}\NormalTok{,}\FloatTok{661.44}\NormalTok{,}\FloatTok{762.1}\NormalTok{,}\FloatTok{200.05}\NormalTok{,}\FloatTok{1705.96}\NormalTok{,}\FloatTok{457.26}\NormalTok{,}\FloatTok{603.5}\NormalTok{,}\FloatTok{1239.68}\NormalTok{,}\FloatTok{3925.64}\NormalTok{)}
\CommentTok{\# Create a data frame using data.frame function on the provided vectors}
\NormalTok{loans\_df }\OtherTok{\textless{}{-}}\FunctionTok{data.frame}\NormalTok{(amount,intRate,loanTerm, loanType,mthPmt)}
\NormalTok{loans\_df}
\end{Highlighting}
\end{Shaded}

\begin{verbatim}
##    amount intRate loanTerm loanType  mthPmt
## 1  200000  0.0700       15    Mortg 1787.23
## 2  150000  0.0750       15    Mortg 1381.88
## 3  100000  0.0700       30    Mortg  661.44
## 4   25000  0.0650        3      Car  762.10
## 5   10000  0.0770        5      Car  200.05
## 6  200000  0.0625       15    Mortg 1705.96
## 7   15000  0.0650        3    Other  457.26
## 8   25000  0.0775        4      Car  603.50
## 9  150000  0.0575       15    Mortg 1239.68
## 10 475000  0.0575       15    Mortg 3925.64
\end{verbatim}

\begin{Shaded}
\begin{Highlighting}[]
\CommentTok{\# Create a vector of column names}
\NormalTok{loan\_col}\OtherTok{\textless{}{-}}\FunctionTok{c}\NormalTok{(}\StringTok{"Amount"}\NormalTok{,}\StringTok{"Rate"}\NormalTok{,}\StringTok{"Term"}\NormalTok{,}\StringTok{"Type"}\NormalTok{,}\StringTok{"Payment"}\NormalTok{)}
\FunctionTok{names}\NormalTok{(loans\_df)}\OtherTok{\textless{}{-}}\NormalTok{loan\_col}


\CommentTok{\# ******************************************************************************}
\CommentTok{\# Slide\#5: Data Frames {-} Basic Attributes}
\CommentTok{\# ******************************************************************************}
\CommentTok{\# Determine the size of the data frame}



\CommentTok{\# Get the structure of the data frame}

\CommentTok{\# Display some and all column names}

\CommentTok{\# Show only the top and bottom portions of a large data frame}

\CommentTok{\# Print the basic summaries of the data in a data frame}


\CommentTok{\# ******************************************************************************}
\CommentTok{\# Slide\#6: Data Frames {-} Rows and Columns}
\CommentTok{\# ******************************************************************************}
\CommentTok{\# Accessing a single column of a data frame}

\CommentTok{\# Notice difference in display, not obvious why}

\CommentTok{\# Accessing consecutive rows and a single column}

\CommentTok{\# Accessing nonconsecutive rows and all columns}
                \CommentTok{\# Experiment with removing the comma}
\CommentTok{\# Accessing columns using their names}
                \CommentTok{\# Experiment with removing the comma}

\CommentTok{\# ******************************************************************************}
\CommentTok{\# Slide\#7: Data Frames {-} Factor Columns}
\CommentTok{\# ******************************************************************************}
\CommentTok{\# Working with factor columns in a data frame}
\CommentTok{\# The type of loan column is not a factor vector}

            \CommentTok{\# data.frame column}
            \CommentTok{\# Double{-}brackets more used with lists}
            \CommentTok{\# character vector}
\CommentTok{\# Parameter drop=FALSE assures data frame type}

\CommentTok{\# Recreating the data frame with loan type as a factor column}


\CommentTok{\# ******************************************************************************}
\CommentTok{\# Slide\#8: Data Frames {-} Indicator Variables}
\CommentTok{\# ******************************************************************************}
\CommentTok{\# Creating indicator (dummy) variables based on different factor levels}
\end{Highlighting}
\end{Shaded}


\end{document}
